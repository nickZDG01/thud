%% Le lingue utilizzate, che verranno passate come opzioni al pacchetto babel. Come sempre, l'ultima indicata sarà quella primaria.
%% Se si utilizzano una o più lingue diverse da "italian" o "english", leggere le istruzioni in fondo.
\def\thudbabelopt{english,italian}
%% Valori ammessi per target: bach (tesi triennale), mst (tesi magistrale), phd (tesi di dottorato).
%% Valori ammessi per aauheader: '' (vuoto -> nessun header Alpen Adria Univeristat), aics (Department of Artificial Intelligence and Cybersecurity), informatics (Department of Informatics Systems). Il nome del dipartimento è allineato con la versione inglese del logo UniUD.
\documentclass[target=mst,aauheader=]{thud}

%% --- Informazioni sulla tesi ---
%% Per tutti i tipi di tesi
% Scommentare quello di interesse, o mettete quello che vi pare
\course{Informatica}
%\course{Internet of Things, Big Data e Web}
%\course{Matematica}
%\course{Comunicazione Multimediale e Tecnologie dell'Informazione}
\title{Vulnerability Assessment di una rete aziendale}
\author{Nicola Zerajic De Giorgio}
\supervisor{Prof.\ Marino Miculan}
\cosupervisor{}
\tutor{}
%% Campi obbligatori: \title, \author e \course.
%% Altri campi disponibili: \reviewer, \tutor, \chair, \date (anno accademico, calcolato in automatico), \rights
%% Con \supervisor, \cosupervisor, \reviewer e \tutor si possono indicare più nomi separati da \and.


%% --- Pacchetti consigliati ---
%% pdfx: per generare il PDF/A per l'archiviazione. Necessario solo per la versione finale
\usepackage[a-1b]{pdfx}
%% hyperref: Regola le impostazioni della creazione del PDF... più tante altre cose. Ricordarsi di usare l'opzione pdfa.
\usepackage[pdfa]{hyperref}
%% tocbibind: Inserisce nell'indice anche la lista delle figure, la bibliografia, ecc.

%% --- Stili di pagina disponibili (comando \pagestyle) ---
%% sfbig (predefinito): Apertura delle parti e dei capitoli col numero grande; titoli delle parti e dei capitoli e intestazioni di pagina in sans serif.
%% big: Come "sfbig", solo serif.
%% plain: Apertura delle parti e dei capitoli tradizionali di LaTeX; intestazioni di pagina come "big".

\begin{document}
\maketitle


%% Indice
\tableofcontents

%% Lista delle tabelle (se presenti)
%\listoftables

%% Lista delle figure (se presenti)
%\listoffigures

%% Corpo principale del documento
\mainmatter

%% Parte
%% La suddivisione in parti è opzionale; solitamente sono sufficienti i capitoli.
%\part{Parte}

%% Capitolo
\chapter{Vulnerability Assessment: cos’è e perché è importante}
Il \textit{vulnerability assessment} (VA) è un processo semi-automatizzato che permette di analizzare un insieme definito di \textit{asset}, ad esempio un’intera \textit{subnet} o un singolo \textit{host} in cerca di debolezze ed errori di configurazione. Bisogna considerare che il VA non è un \textit{penetration test}\footnote{Attività, spesso manuale e poco economica, che mira ad attaccare, come farebbe un malintenzionato, un’infrastruttura di rete. Di solito è un’attività che si svolge sporadicamente o al termine della progettazione di una rete aziendale per testarne la sicurezza e i sistemi di rilevamento intrusione e di emergenza.}. È invece un’attività che si dovrebbe svolgere periodicamente durante l’anno, sia per validare eventuali modifiche all’infrastruttura informatica, ad esempio aggiunta o riconfigurazione di apparati di rete o postazioni di lavoro, sia per verificare che il sistema non sia affetto dalle ultime vulnerabilità pubblicate dagli enti di ricerca\footnote{Ad esempio il sito del National Institute of Standards and Technology (NIST) pubblica periodicamente nuove vulnerabilità sul National Vulnerability Database.} e dai produttori. 
Lo scopo di un VA, quindi, è quello di identificare, quantificare, classificare e prioritizzare i possibili difetti di un sistema informatico, ovvero le sue vulnerabilità. Come afferma il SysAdmin, Audit, Networking and Security Institute\footnote{Il SANS Institute, fondato negli Stati Uniti nel 1989, fornisce percorsi di educazione e addestramento in materia di sicurezza informatica.} \textit{«vulnerabilities are the gateways by which threats are manifested»}. Le vulnerabilità sono quindi le debolezze attraverso le quali un sistema può essere compromesso.
Nella maggior parte dei casi le vulnerabilità di un sistema, ad esempio una postazione di lavoro, sono conseguenza di difetti o errori di programmazione, i cosiddetti \textit{bug}, del sistema operativo (o del firmware) e dei software installati su di esso. Spesso le case di sviluppo, nel processo di progettazione e scrittura di un software, prediligono un approccio orientato al raggiungimento di obiettivi quali implementazione di nuove \textit{features}, usabilità, performance e soprattutto costi bassi e tempi rapidi. Tutto ciò si traduce in software con bassi standard di sicurezza e con la necessità da parte degli utenti di eseguire periodici aggiornamenti di sicurezza, soprattutto nelle fasi iniziali del rilascio del software. Tuttavia, anche chi segue un approccio di \textit{security by design}\footnote{Approccio di sviluppo software e hardware che considera la sicurezza come requisito principale, adattando il resto della progettazione ad essa.} nello sviluppo delle proprie applicazioni non è esente da una periodica manutenzione e da un’accurata analisi delle possibili falle di sicurezza nel codice. Una delle preoccupazioni più grandi di uno sviluppatore, infatti, è quella di vedere pubblicato sulla rete, solitamente nel \textit{Dark Web}, un \textit{exploit}\footnote{Procedura (spesso automatizzata in forma di script) che permette di violare un sistema informatico sfruttando una sua vulnerabilità.} di una vulnerabilità zero-day di un proprio programma. Tali vulnerabilità sono molto pericolose in quanto al momento della loro scoperta e pubblicazione non sono ancora disponibili \textit{patch} di sicurezza.
A seguito di queste considerazioni e tenendo conto del grande numero di applicazioni e servizi installati su una semplice postazione di lavoro, si può immaginare quanto sia estesa la superficie di attacco di un’infrastruttura informatica aziendale. Ogni software è una potenziale porta aperta per un criminale. Risulta quindi fondamentale avere uno strumento come il \textit{vulnerability assessment} che permetta di rilevare in anticipo e prima di un possibile malintenzionato le falle di sicurezza più critiche della rete aziendale, così da prevenire conseguenze potenzialmente disastrose.
Le debolezze di un’infrastruttura informatica però non si limitano agli applicativi software. Una corretta e robusta configurazione degli apparati di rete, quali switch, router, firewall, server e NAS, contribuisce a rendere l’infrastruttura più resistente e meno violabile. Nel migliore degli scenari, infatti, avere opportune regole che permettano di organizzare, limitare e controllare sia le connessioni tra i sistemi che i singoli accessi degli utenti fisici a tali apparati, consente di isolare gli attacchi e limitare quindi i danni. Il VA aiuta anche in questo senso ad irrobustire la struttura informatica aziendale andando ad analizzare, rilevare e dare delle possibili soluzioni ad eventuali errori di configurazione e sviste degli amministratori di sistema, dall’uso di password di accesso deboli o di default, all’impiego di protocolli insicuri e obsoleti, all’esposizione in rete di servizi non essenziali.
Il VA si rivela essere quindi il primo strumento di protezione e di messa in sicurezza delle risorse e degli asset aziendali. Esso si sviluppa in più fasi: pianificazione e raccolta delle informazioni riguardo l’organizzazione e l’infrastruttura aziendale, esecuzione della scansione, analisi dei risultati, risoluzione delle problematiche identificate o \textit{remediation}.


%% Sezione
\section{Titolo della Sezione}
Donec pulvinar neque non lectus vulputate pellentesque. Quisque rutrum arcu velit, in feugiat sapien posuere vel. Praesent metus orci, aliquam ac cursus eget, fermentum a nisl. Etiam eu augue lacus. Nam nisi sapien, mattis sed vehicula non, pellentesque at quam. Sed euismod, dolor nec commodo lobortis, erat erat ultricies eros, bibendum dictum nulla felis in dui. Nulla blandit ultrices arcu, vitae lacinia tellus tempor sit amet. Nulla non tincidunt dolor. In eget luctus sem, sed elementum ligula. Proin elementum adipiscing sem, sit amet ultricies nisl tincidunt eu. Ut lobortis dui quam, et scelerisque erat ultrices sit amet. Sed libero sem, mollis quis euismod quis, suscipit ac justo.

%% Sottosezione
\subsection{Sottosezione}
Donec cursus tortor eget sem ornare imperdiet. Ut vel orci non ipsum condimentum laoreet vitae ut sapien. Aenean metus mi, vehicula quis turpis nec, porttitor blandit dui. Nullam sed sollicitudin quam. Fusce nisl ante, commodo eget lacus ac, mollis ullamcorper neque. Quisque faucibus dictum nisl, dignissim fermentum sapien fringilla vel. Proin dui velit, molestie sit amet sapien et, pellentesque tristique purus. Curabitur ac quam ac diam varius bibendum.

Lorem ipsum dolor sit amet, consectetur adipiscing elit. Aliquam auctor odio sit amet tempor consequat. Suspendisse auctor vel augue ut aliquet. Suspendisse potenti. Nullam eget leo eu neque ornare mattis in in tortor. Phasellus finibus, orci nec interdum placerat, leo libero rutrum ipsum, vel varius elit leo vel lacus. Pellentesque a tristique nibh, dictum egestas augue. Aliquam egestas ligula nec diam vulputate, vel bibendum justo sodales. Nunc imperdiet ut est non ultrices. Pellentesque enim tortor, suscipit et neque vel, varius gravida dolor. In sed lobortis tellus. Duis ac sem ac mi aliquet imperdiet eu eget ligula. Aliquam ligula nisl, rhoncus eget orci a, cursus dapibus urna. Donec commodo consequat mi.

Donec elementum erat vitae nunc commodo ultrices. Vivamus auctor arcu pharetra orci cursus, sodales dictum urna egestas. Pellentesque eget felis augue. Quisque tempus non nulla sed aliquam. Nam sagittis eros lectus, a mattis felis malesuada at. Sed et semper dui. Nunc luctus, diam ac ultricies euismod, augue odio consequat ante, vel vehicula ipsum mauris sed nunc. Pellentesque consequat, eros et volutpat maximus, justo metus commodo ante, vitae egestas nisl mi non orci.

Vestibulum aliquet tempor magna sit amet tincidunt. Vivamus congue leo eu lectus vestibulum, ut tincidunt tellus viverra. Etiam sit amet blandit dolor. Sed eget varius dolor. Cras porta euismod mi feugiat posuere. Proin elit nulla, pulvinar vitae fermentum id, varius nec leo. Nullam eu erat nulla. Etiam erat ante, lobortis vel sagittis quis, pretium in lectus. Aenean accumsan, quam sagittis fringilla gravida, nulla massa ullamcorper nisi, non pretium leo magna ac tellus. Nunc vel lectus dapibus, pulvinar sem id, gravida urna. Duis tempor imperdiet consectetur. Quisque nec risus diam.

Pellentesque purus dui, lobortis sed maximus at, consequat eu ex. Phasellus eu porttitor dolor. Integer ornare justo elit, eget porttitor tortor porta vel. Quisque sit amet urna tempor, mollis quam in, faucibus orci. Sed auctor efficitur malesuada. Aliquam lacinia laoreet augue, varius imperdiet sem bibendum at. Mauris neque nunc, fringilla ac commodo mollis, pellentesque tincidunt est.

Nunc commodo, quam id maximus dignissim, ligula magna convallis mauris, a elementum nisl metus at nisl. Donec risus ante, maximus at iaculis eget, placerat nec nulla. Sed tincidunt sodales justo eget finibus. Sed urna magna, dapibus at massa scelerisque, facilisis fringilla justo. Pellentesque habitant morbi tristique senectus et netus et malesuada fames ac turpis egestas. Nulla eleifend semper erat, ut placerat metus. Quisque laoreet ultrices est. Sed porta fermentum odio. Cras ornare erat nec ultricies hendrerit.

Nunc commodo, libero consectetur imperdiet faucibus, nisi mauris pulvinar quam, pellentesque vehicula neque mi maximus justo. Nulla tristique, lacus et rhoncus varius, orci nibh dictum mauris, fringilla elementum elit elit ut tortor. Curabitur euismod quis sem et mollis. Sed feugiat dignissim tincidunt. Mauris cursus dictum venenatis. Nunc fermentum at metus ac vehicula. Duis sodales tellus magna, sit amet luctus urna imperdiet sed.

Fusce dolor urna, suscipit sed mi aliquam, aliquam convallis leo. Pellentesque quam nunc, consequat non posuere sed, fermentum ut urna. Nullam vehicula rutrum ex, at vehicula sem rhoncus a. Vivamus scelerisque, dolor in tristique mollis, leo diam laoreet velit, at semper enim velit ac dui. Donec placerat, odio nec maximus ultricies, tellus nibh lobortis orci, ut gravida ligula nibh in ante. Aenean blandit viverra tempus. Class aptent taciti sociosqu ad litora torquent per conubia nostra, per inceptos himenaeos. In vitae efficitur est. Nam venenatis, nisi vitae condimentum pretium, nunc libero accumsan nulla, in accumsan turpis enim at velit. Donec semper dolor posuere orci facilisis volutpat. Vestibulum lorem leo, tempor vitae porta id, mattis commodo dui. Cras aliquet non risus at accumsan. Suspendisse viverra, mi in blandit hendrerit, enim magna faucibus eros, id egestas lacus magna non leo.

Morbi suscipit tempor ante a varius. Fusce eu lobortis eros. Quisque quis pharetra metus, dictum luctus ligula. In a tincidunt nibh, vel scelerisque arcu. Donec consequat justo condimentum tempus iaculis. Nullam interdum id metus vel sagittis. Proin ante tortor, posuere a vehicula sit amet, dapibus ut metus. Etiam consectetur, tortor sit amet lacinia varius, nibh purus vehicula ex, sit amet fermentum justo urna a ante.

Phasellus feugiat neque sed risus faucibus tristique nec sit amet magna. Fusce lacus erat, fermentum non pharetra et, auctor et tellus. Donec gravida id urna quis auctor. Quisque ultrices id orci id posuere. In hac habitasse platea dictumst. Phasellus ante nibh, fermentum a vehicula sit amet, efficitur vel orci. Phasellus eget justo orci. Fusce ut tellus nisl. Nullam sit amet ipsum justo. Proin suscipit purus facilisis, tristique arcu in, venenatis dui. Quisque a pellentesque augue. Sed blandit viverra feugiat. Curabitur volutpat luctus maximus. Nulla sed condimentum nibh, sed rutrum velit. Cras pellentesque lectus est, vestibulum vehicula arcu elementum a.

Sed dapibus augue rutrum augue ultrices, in tristique lectus malesuada. Praesent pharetra at arcu eu pellentesque. Cras ac est arcu. Suspendisse euismod pellentesque diam, id porta metus. Donec porta ex vel ornare tincidunt. Nunc non tortor sodales, lacinia diam quis, malesuada lectus. Cras a ultrices massa, a semper ligula.




%% Fine dei capitoli normali, inizio dei capitoli-appendice (opzionali)
\appendix

%\part{Appendici}

\chapter{Titolo della prima appendice}
Sed purus libero, vestibulum ut nibh vitae, mollis ultricies augue. Pellentesque velit libero, tempor sed pulvinar non, fermentum eu leo. Duis posuere eleifend nulla eget sagittis. Nam laoreet accumsan rutrum. Interdum et malesuada fames ac ante ipsum primis in faucibus. Curabitur eget libero quis leo porttitor vehicula eget nec odio. Proin euismod interdum ligula non ultricies. Maecenas sit amet accumsan sapien.

%% Parte conclusiva del documento; tipicamente per riassunto, bibliografia e/o indice analitico.
\backmatter

%% Riassunto (opzionale)
%\summary
%Maecenas tempor elit sed arcu commodo, dapibus sagittis leo egestas. Praesent at ultrices urna. Integer et nibh in augue mollis facilisis sit amet eget magna. Fusce at porttitor sapien. Phasellus imperdiet, felis et molestie vulputate, mauris sapien tincidunt justo, in lacinia velit nisi nec ipsum. Duis elementum pharetra lorem, ut pellentesque nulla congue et. Sed eu venenatis tellus, pharetra cursus felis. Sed et luctus nunc. Aenean commodo, neque a aliquam bibendum, mauris augue fringilla justo, et scelerisque odio mi sit amet diam. Nulla at placerat nibh, nec rutrum urna. Donec ut egestas magna. Aliquam erat volutpat. Phasellus vestibulum justo sed purus mattis, vitae lacinia magna viverra. Nulla rutrum diam dui, vel semper mi mattis ac. Vestibulum ante ipsum primis in faucibus orci luctus et ultrices posuere cubilia Curae; Donec id vestibulum lectus, eget tristique est.

%% Bibliografia (praticamente obbligatoria)
\bibliographystyle{plain_\languagename}%% Carica l'omonimo file .bst, dove \languagename è la lingua attiva.
%% Nel caso in cui si usi un file .bib (consigliato)
\bibliography{thud}
%% Nel caso di bibliografia manuale, usare l'environment thebibliography.

%% Per l'indice analitico, usare il pacchetto makeidx (o analogo).

\end{document}

--- Istruzioni per l'aggiunta di nuove lingue ---
Per ogni nuova lingua utilizzata aggiungere nel preambolo il seguente spezzone:
    \addto\captionsitalian{%
        \def\abstractname{Sommario}%
        \def\acknowledgementsname{Ringraziamenti}%
        \def\authorcontactsname{Contatti dell'autore}%
        \def\candidatename{Candidato}%
        \def\chairname{Direttore}%
        \def\conclusionsname{Conclusioni}%
        \def\cosupervisorname{Co-relatore}%
        \def\cosupervisorsname{Co-relatori}%
        \def\cyclename{Ciclo}%
        \def\datename{Anno accademico}%
        \def\indexname{Indice analitico}%
        \def\institutecontactsname{Contatti dell'Istituto}%
        \def\introductionname{Introduzione}%
        \def\prefacename{Prefazione}%
        \def\reviewername{Controrelatore}%
        \def\reviewersname{Controrelatori}%
        %% Anno accademico
        \def\shortdatename{A.A.}%
        \def\summaryname{Riassunto}%
        \def\supervisorname{Relatore}%
        \def\supervisorsname{Relatori}%
        \def\thesisname{Tesi di \expandafter\ifcase\csname thud@target\endcsname Laurea\or Laurea Magistrale\or Dottorato\fi}%
        \def\tutorname{Tutor aziendale%
        \def\tutorsname{Tutor aziendali}%
    }
sostituendo a "italian" (nella 1a riga) il nome della lingua e traducendo le varie voci.
